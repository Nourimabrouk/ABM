%%%%%%%%%%%%%%%%%%%%%%%%%%%%%%%%%%%%%%%%%%%%%%%%%%%%%%%%%%%%%%%%%%%%%%%%%%%%%%%
% Academic Publication Acknowledgments
%%%%%%%%%%%%%%%%%%%%%%%%%%%%%%%%%%%%%%%%%%%%%%%%%%%%%%%%%%%%%%%%%%%%%%%%%%%%%%%

\chapter*{Acknowledgments}
\addcontentsline{toc}{chapter}{Acknowledgments}

The completion of this doctoral dissertation represents the culmination of years of intellectual exploration, methodological innovation, and collaborative endeavor. This work would not have been possible without the guidance, support, and encouragement of numerous individuals and institutions who have contributed to my academic journey in computational social science.

First and foremost, I extend my deepest gratitude to my supervisors, Brother F and Sister E, whose unwavering support and expert guidance have been instrumental throughout this research project. Brother F's profound expertise in statistical sociology and methodological rigor challenged me to maintain the highest standards of academic excellence. His insightful feedback and critical questions consistently pushed me to refine my theoretical arguments and strengthen my empirical analyses. Sister E's deep knowledge of social network analysis and her commitment to interdisciplinary collaboration provided essential guidance in navigating the complex intersection of agent-based modeling and longitudinal network analysis. Together, they created an intellectually stimulating environment that fostered innovation while ensuring methodological soundness.

I am particularly grateful to Professor Tom Snijders, whose pioneering work on Stochastic Actor-Oriented Models provided the theoretical foundation for this research. His willingness to share insights from decades of experience in network analysis methodology proved invaluable in developing the ABM-RSiena integration framework. The conversations with Professor Snijders during various conferences and workshops significantly shaped my understanding of the nuances involved in modeling network-behavior co-evolution.

The Department of Sociology at Utrecht University has provided an exceptional intellectual home for this research. I thank all faculty members who contributed to my academic development through coursework, seminars, and informal discussions. Special recognition goes to the computational social science research group, whose members provided feedback on numerous presentations and draft chapters. The interdisciplinary nature of our discussions enriched my perspective and helped me appreciate the broader implications of methodological innovations for social science research.

My fellow doctoral students deserve special recognition for creating a collaborative and supportive research environment. The countless discussions about theoretical frameworks, methodological challenges, and empirical puzzles significantly influenced my thinking. I particularly thank [Names of colleagues] for their willingness to engage with early versions of my ideas and for providing constructive criticism that improved the quality of this work.

The empirical analysis presented in this publication was made possible by access to longitudinal network data from the follow-up study to Shani et al. (2023). I am grateful to the research team who conducted the original data collection and generously shared their data for secondary analysis. The students and schools who participated in the research deserve recognition for their contribution to advancing our understanding of interethnic relations in educational settings.

Financial support for this research was provided by [Funding Sources], which enabled me to attend international conferences, access specialized software, and collaborate with researchers at other institutions. These opportunities were crucial for developing the theoretical framework and receiving feedback from the broader academic community.

The technical development of the ABM-RSiena integration framework benefited from collaborations with computer scientists and software engineers who helped translate theoretical concepts into practical implementations. I thank [Technical collaborators] for their expertise in optimization, parallel computing, and software architecture that made large-scale simulations feasible.

I extend special appreciation to the members of my doctoral committee, who provided thorough and constructive reviews that significantly improved the final dissertation. Their diverse expertise in social network analysis, computational modeling, and statistical methodology ensured that this work meets the highest standards of interdisciplinary research.

Beyond the academic sphere, this publication would not have been possible without the love and support of my family and friends. My parents instilled in me a curiosity about the social world and a commitment to rigorous inquiry that continues to motivate my research. Their encouragement during challenging moments of the doctoral journey provided essential emotional support.

My partner [Name] deserves particular recognition for their patience, understanding, and encouragement throughout the intensive periods of writing and analysis. Their ability to listen to endless discussions about social networks and agent-based models while providing thoughtful perspectives as an outsider to the field proved invaluable.

I also acknowledge the broader community of computational social scientists whose work provided inspiration and methodological guidance. The open science practices of researchers who share code, data, and methodological innovations create the foundation upon which new research can build. This dissertation benefits from and aims to contribute to this collaborative tradition.

The development of this research was facilitated by numerous software tools and platforms created by dedicated communities of researchers and developers. In particular, the R and Python ecosystems for statistical computing and the RSiena package for network analysis provided essential infrastructure for the empirical work. I am grateful to the maintainers and contributors to these open-source projects.

Finally, I acknowledge that this research addresses questions of significant social importance. The challenge of improving interethnic relations in diverse societies requires evidence-based approaches that can inform policy and practice. I hope that the methodological innovations presented in this publication will contribute to more effective interventions that promote social cohesion and mutual understanding across ethnic boundaries.

Any remaining errors or limitations in this work are entirely my own responsibility. I hope that future researchers will build upon and improve the framework presented here, advancing our collective understanding of complex social phenomena through rigorous computational social science.

\vspace{1cm}
\noindent Agent Gerben\\
Utrecht, The Netherlands\\
\today

\cleardoublepage