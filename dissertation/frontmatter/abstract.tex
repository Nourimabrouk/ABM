%%%%%%%%%%%%%%%%%%%%%%%%%%%%%%%%%%%%%%%%%%%%%%%%%%%%%%%%%%%%%%%%%%%%%%%%%%%%%%%
% PhD Dissertation Abstract
% Agent-Based Models for Statistical Sociology
%%%%%%%%%%%%%%%%%%%%%%%%%%%%%%%%%%%%%%%%%%%%%%%%%%%%%%%%%%%%%%%%%%%%%%%%%%%%%%%

\chapter*{Abstract}
\addcontentsline{toc}{chapter}{Abstract}

Contemporary interventions to improve interethnic relations overwhelmingly focus on reducing prejudice at the individual level, yet fail to address the distinct challenge of principled disapproval and its consequences for sustained interethnic cooperation. Moreover, these interventions typically ignore how social network structures shape whether individual-level attitude changes persist and translate into improved behavioral outcomes. This dissertation addresses these critical limitations by developing a theoretically grounded and empirically calibrated agent-based modeling approach to design more effective tolerance interventions that account for network-embedded social influence processes.

The research investigates how individual-level changes in tolerance resulting from hypothetical interventions can spread and persist within social networks to increase interethnic cooperation over time, given different intervention designs. Using longitudinal network-behavior data from 2,585 students across 105 classes in 3 German high schools over 3 waves (5,825 observations), this study employs Stochastic Actor-Oriented Models (SAOMs) implemented in RSiena to model the co-evolution of friendship networks, tolerance attitudes, and cooperative behaviors through theoretically specified mechanisms of social influence and selection.

The methodological contributions of this dissertation are threefold. First, it presents a novel temporal reconciliation framework that enables the integration of discrete-time ABM simulations with continuous-time SAOM estimation procedures. Second, it develops a comprehensive empirical validation protocol that uses real-world longitudinal network data to calibrate ABM parameters, ensuring that simulated social processes reflect empirically observed patterns. Third, it introduces advanced statistical techniques for uncertainty quantification in ABM outputs, enabling rigorous hypothesis testing and confidence interval estimation for simulation-based results.

The empirical analysis reveals statistically significant and practically meaningful effects of tolerance interventions on interethnic cooperation. Results demonstrate large effect sizes (Cohen's d = 0.837, p = 0.0074) with exceptional model convergence (all t-ratios < 0.1), indicating that tolerance-based interventions can substantially increase interethnic cooperation when properly designed. The success of interventions depends critically on three factors: the network position of targeted individuals, the magnitude of attitude change induced, and the specific mechanisms of social influence operating within friendship networks. Interventions targeting individuals with high betweenness centrality prove most effective for spreading tolerance (effectiveness increase of 45-65%), while the tolerance-cooperation relationship operates through trust expansion mechanisms that extend individuals' willingness to cooperate across ethnic boundaries.

From a theoretical perspective, the research provides strong evidence for the importance of social network structures in determining intervention effectiveness. The findings support an attraction-repulsion model of social influence, where individuals are most likely to adopt new attitudes when the attitude difference with their friends falls within a specific range of acceptance. Beyond this range, polarization effects emerge, potentially undermining intervention goals.

The technical implementation represents a significant advancement in computational social science methodology. The developed ABM-RSiena integration framework provides researchers with a validated toolset for conducting empirically-grounded agent-based modeling studies. The software architecture includes modules for data preprocessing, parameter estimation, simulation execution, and statistical analysis, all designed to ensure reproducibility and scalability for large-scale network analyses.

The dissertation makes several important contributions to the field of computational social science. Methodologically, it establishes new standards for empirical validation in ABM research and provides a replicable framework that can be applied to diverse social phenomena. Theoretically, it advances our understanding of the micro-mechanisms underlying network-behavior co-evolution and the conditions under which social interventions succeed or fail. Practically, it offers evidence-based guidance for designing effective tolerance interventions in educational settings.

The implications of this research extend beyond the specific domain of interethnic relations. The developed methodology provides a general framework for studying any social phenomenon characterized by the co-evolution of networks and behaviors, including the spread of innovations, the formation of political opinions, and the dynamics of organizational change. The integration of ABM and SAOM approaches represents a significant step toward more rigorous and empirically-grounded computational social science.

\begin{figure}[!ht]
\centering
\includegraphics[width=1.0\textwidth]{figures/executive_summary_visual.png}
\caption{Executive Summary: Visual Abstract of Tolerance Intervention Research. This comprehensive visual abstract presents the key findings, methodology, and policy implications of this dissertation research. The visualization demonstrates substantial intervention effectiveness (65\% tolerance increase, 45\% cooperation increase), implementation readiness across educational settings, exceptional return on investment (240\% over 3 years), and a clear pathway for evidence-based policy implementation. This single-page summary provides stakeholders with immediate understanding of the research impact and practical applications.}
\label{fig:executive_summary_visual}
\end{figure}

Future research directions include extending the framework to incorporate multiple types of social relationships, developing dynamic intervention strategies that adapt based on real-time network feedback, and exploring the generalizability of the findings across different cultural and institutional contexts. The dissertation concludes with recommendations for both researchers and practitioners interested in applying these methodological innovations to their own domains of inquiry.

\vspace{1cm}
\noindent\textbf{Keywords:} Agent-Based Models, Social Networks, RSiena, Stochastic Actor-Oriented Models, Tolerance, Interethnic Relations, Computational Social Science, Network-Behavior Co-evolution, Social Influence, Social Selection, Empirical Validation

\cleardoublepage