%%%%%%%%%%%%%%%%%%%%%%%%%%%%%%%%%%%%%%%%%%%%%%%%%%%%%%%%%%%%%%%%%%%%%%%%%%%%%%%
% PhD Dissertation Abstract
% Agent-Based Models for Statistical Sociology
%%%%%%%%%%%%%%%%%%%%%%%%%%%%%%%%%%%%%%%%%%%%%%%%%%%%%%%%%%%%%%%%%%%%%%%%%%%%%%%

\chapter*{Abstract}
\addcontentsline{toc}{chapter}{Abstract}

Agent-Based Models (ABMs) have emerged as a powerful methodological approach in computational social science, offering unique capabilities for understanding complex social phenomena through the simulation of individual-level interactions and their emergent macro-level outcomes. However, a critical limitation in current ABM research is the lack of robust empirical validation frameworks that can bridge the gap between theoretical model specifications and real-world social processes. This dissertation addresses this methodological challenge by developing and implementing an innovative integration framework that combines Agent-Based Modeling with Stochastic Actor-Oriented Models (SAOMs) as implemented in RSiena.

The research focuses on a specific application domain: understanding how tolerance-based interventions designed to improve interethnic relations can spread and persist within social networks, ultimately affecting interethnic cooperation over time. Using longitudinal network data from German high schools, this study develops a comprehensive theoretical framework that models the co-evolution of social networks and individual attitudes through mechanisms of social influence and social selection.

The methodological contributions of this dissertation are threefold. First, it presents a novel temporal reconciliation framework that enables the integration of discrete-time ABM simulations with continuous-time SAOM estimation procedures. Second, it develops a comprehensive empirical validation protocol that uses real-world longitudinal network data to calibrate ABM parameters, ensuring that simulated social processes reflect empirically observed patterns. Third, it introduces advanced statistical techniques for uncertainty quantification in ABM outputs, enabling rigorous hypothesis testing and confidence interval estimation for simulation-based results.

The empirical analysis reveals several key findings regarding the effectiveness of different intervention designs. The results demonstrate that the success of tolerance-based interventions depends critically on three factors: the network position of targeted individuals, the magnitude of attitude change induced by the intervention, and the specific mechanisms of social influence operating within the network. Interventions targeting individuals with high betweenness centrality prove most effective for spreading tolerance throughout the network, while the relationship between tolerance and interethnic cooperation is mediated by changes in interpersonal trust and perceived social norms.

From a theoretical perspective, the research provides strong evidence for the importance of social network structures in determining intervention effectiveness. The findings support an attraction-repulsion model of social influence, where individuals are most likely to adopt new attitudes when the attitude difference with their friends falls within a specific range of acceptance. Beyond this range, polarization effects emerge, potentially undermining intervention goals.

The technical implementation represents a significant advancement in computational social science methodology. The developed ABM-RSiena integration framework provides researchers with a validated toolset for conducting empirically-grounded agent-based modeling studies. The software architecture includes modules for data preprocessing, parameter estimation, simulation execution, and statistical analysis, all designed to ensure reproducibility and scalability for large-scale network analyses.

The dissertation makes several important contributions to the field of computational social science. Methodologically, it establishes new standards for empirical validation in ABM research and provides a replicable framework that can be applied to diverse social phenomena. Theoretically, it advances our understanding of the micro-mechanisms underlying network-behavior co-evolution and the conditions under which social interventions succeed or fail. Practically, it offers evidence-based guidance for designing effective tolerance interventions in educational settings.

The implications of this research extend beyond the specific domain of interethnic relations. The developed methodology provides a general framework for studying any social phenomenon characterized by the co-evolution of networks and behaviors, including the spread of innovations, the formation of political opinions, and the dynamics of organizational change. The integration of ABM and SAOM approaches represents a significant step toward more rigorous and empirically-grounded computational social science.

Future research directions include extending the framework to incorporate multiple types of social relationships, developing dynamic intervention strategies that adapt based on real-time network feedback, and exploring the generalizability of the findings across different cultural and institutional contexts. The dissertation concludes with recommendations for both researchers and practitioners interested in applying these methodological innovations to their own domains of inquiry.

\vspace{1cm}
\noindent\textbf{Keywords:} Agent-Based Models, Social Networks, RSiena, Stochastic Actor-Oriented Models, Tolerance, Interethnic Relations, Computational Social Science, Network-Behavior Co-evolution, Social Influence, Social Selection, Empirical Validation

\cleardoublepage