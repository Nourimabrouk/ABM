\chapter{Introduction}

\section{The Challenge of Interethnic Cooperation}

In an increasingly diverse world, the promotion of interethnic cooperation has become one of the most pressing challenges of our time. From neighborhood communities to multinational organizations, the ability of individuals from different ethnic backgrounds to work together effectively determines the success of social integration and the flourishing of multicultural societies. Yet despite decades of research and intervention efforts, sustainable interethnic cooperation remains elusive in many contexts.

Traditional approaches to promoting interethnic cooperation have predominantly focused on reducing prejudice through intergroup contact interventions \citep{allport1954nature}. The underlying logic is straightforward: if we can reduce negative attitudes toward outgroups, cooperative behavior will naturally follow. However, mounting evidence suggests that this prejudice-reduction paradigm has significant limitations. Meta-analyses reveal that while contact interventions can reduce explicit prejudice under optimal conditions, their effects on actual behavior are often modest and fail to persist over time \citep{paluck2019contact}.

The limitations of prejudice-focused interventions become particularly apparent when considering the complexity of real-world interethnic relations. Individuals may hold principled objections to certain cultural practices or hold different values while still being willing to cooperate across ethnic lines. In such cases, the goal should not necessarily be to change minds or eliminate all disagreement, but rather to foster tolerance—the willingness to accept and cooperate with others despite disapproval or disagreement \citep{verkuyten2020tolerance}.

\section{Research Objectives}

This dissertation addresses a fundamental question: \textbf{How can tolerance interventions promote sustained interethnic cooperation in diverse social networks?} This central question guides an investigation that bridges social psychology, network science, and computational sociology to develop evidence-based strategies for fostering tolerance and cooperation.

To address this overarching question, this research pursues three specific objectives:

\begin{enumerate}
\item \textbf{Understand tolerance diffusion mechanisms}: How does tolerance spread through social networks, and what network and individual characteristics facilitate or impede this process?

\item \textbf{Optimize intervention design}: What intervention characteristics (targeting strategy, dosage, timing) maximize the promotion of interethnic cooperation through tolerance?

\item \textbf{Identify moderating factors}: How do network structures, social contexts, and individual differences moderate the effectiveness of tolerance interventions?
\end{enumerate}

These objectives are pursued through an innovative methodological approach that combines Stochastic Actor-Oriented Models (SAOMs) with agent-based modeling to create a forecasting framework for intervention design. This represents a significant departure from traditional experimental approaches that test interventions after implementation, instead enabling researchers to optimize intervention strategies before deployment.

\section{Methodological Innovation}

This research introduces several methodological innovations that advance the field of computational social science and network intervention research:

\textbf{Predictive Network Modeling}: Rather than simply describing network dynamics post-hoc, this study uses Stochastic Actor-Oriented Models as a forecasting tool to predict the effects of different intervention strategies before implementation. This approach transforms SAOMs from descriptive to prescriptive tools, enabling evidence-based intervention optimization.

\textbf{Custom RSiena Effects}: The research develops novel statistical effects within the RSiena framework to capture attraction-repulsion dynamics in social influence. These effects formalize how individuals are influenced by others who are similar (attraction) while potentially diverging from those who are too different (repulsion), providing a more nuanced understanding of social influence than traditional linear models.

\textbf{Complex Contagion Framework}: Moving beyond simple contagion models that assume influence from any connected individual, this research implements complex contagion mechanisms where influence requires multiple exposures or specific network configurations. This approach better captures how tolerance—as a complex social norm—spreads through networks.

\textbf{Multi-level Integration}: The research integrates individual-level psychological processes (attitude change), dyadic-level social processes (relationship formation), and network-level structural processes (influence diffusion) within a single coherent framework. This multi-level approach enables examination of how macro-level intervention strategies produce changes through micro-level mechanisms.

\section{Theoretical Contributions}

This dissertation makes several theoretical contributions to our understanding of tolerance, social influence, and interethnic cooperation:

\textbf{Tolerance as Behavioral Strategy}: The research develops a conceptualization of tolerance not merely as an attitude, but as a behavioral strategy that enables cooperation despite disagreement. This perspective shifts focus from changing minds to changing behaviors, opening new avenues for intervention.

\textbf{Network Theory of Social Norms}: The study contributes to network theories of social norms by examining how tolerance norms emerge, spread, and stabilize in social networks. This includes identifying the conditions under which tolerance norms can become self-sustaining without continued intervention.

\textbf{Attraction-Repulsion Influence}: The research formalizes attraction-repulsion dynamics in social influence, providing theoretical and empirical foundation for understanding when social influence promotes convergence versus divergence in attitudes and behaviors.

\section{Practical Implications}

Beyond theoretical contributions, this research has significant practical implications for intervention design and social policy:

\textbf{Evidence-Based Targeting}: The predictive modeling approach enables practitioners to identify optimal targets for tolerance interventions before implementation, potentially increasing effectiveness while reducing costs.

\textbf{Intervention Timing}: The research provides insights into when tolerance interventions are most likely to succeed, including the identification of network states and social contexts that are most conducive to positive change.

\textbf{Scalability Assessment}: The modeling framework enables assessment of how tolerance interventions might scale from small groups to larger populations, addressing a critical gap in intervention research.

\textbf{Policy Recommendations}: The research generates specific, actionable recommendations for schools, organizations, and communities seeking to promote interethnic cooperation through tolerance-based approaches.

\section{Dissertation Structure}

This dissertation is organized into eight chapters that systematically build toward answering the central research question:

\textbf{Chapter 2} provides a comprehensive review of relevant literature, synthesizing research on intergroup contact, tolerance, social influence, and network interventions. This chapter establishes the theoretical foundation and identifies gaps that this research addresses.

\textbf{Chapter 3} develops the theoretical framework, formally defining key concepts (tolerance, cooperation, friendship) and specifying the mechanisms through which tolerance interventions are hypothesized to promote interethnic cooperation.

\textbf{Chapter 4} details the methodology, including the development of custom RSiena effects, the integration of SAOMs with simulation methods, and the design of the intervention experiments.

\textbf{Chapter 5} presents the empirical analysis of baseline network and behavioral dynamics using three waves of data from 2,585 students in 105 classrooms across three schools.

\textbf{Chapter 6} reports the results of extensive simulation experiments testing different tolerance intervention strategies under varying conditions.

\textbf{Chapter 7} discusses the implications of the findings for theory and practice, addressing limitations and suggesting directions for future research.

\textbf{Chapter 8} concludes with specific recommendations for tolerance intervention design and implementation, as well as broader implications for computational social science methodology.

The appendices provide technical details of the statistical models, supplementary analyses, and replication materials to support transparency and reproducibility.

Through this systematic investigation, the dissertation aims to advance both theoretical understanding of tolerance and social influence and practical capabilities for designing effective interventions to promote interethnic cooperation in diverse societies.