\chapter{A Micro-Theory of Tolerance and Cooperation}

\section{Introduction}

This chapter develops a comprehensive theoretical framework linking tolerance to interethnic cooperation through social influence processes in networks. The framework addresses three fundamental questions: (1) How do tolerance attitudes change through social influence? (2) How does tolerance affect willingness to form cooperative relationships? (3) How do network structures moderate these processes? The theory integrates insights from social psychology, network science, and agent-based modeling to create a foundation for predictive intervention modeling.

\section{Core Concepts and Definitions}

\subsection{Tolerance}

For this research, tolerance is defined as \textbf{the behavioral intention to refrain from interfering with others' practices or expressions despite personal disapproval or disagreement}. This definition emphasizes several key features:

\begin{itemize}
\item \textbf{Behavioral Focus}: Tolerance is measured through intended or actual behavior rather than attitudes alone
\item \textbf{Principled Disagreement}: Tolerance explicitly acknowledges disagreement or disapproval rather than requiring attitude change
\item \textbf{Non-Interference}: The core behavioral component is refraining from interference rather than requiring active support
\item \textbf{Voluntary Choice**: Tolerance represents a deliberate choice rather than passive indifference
\end{itemize}

This conceptualization differs fundamentally from related concepts such as acceptance (which implies agreement or approval) or respect (which implies positive evaluation). The key distinction is that tolerance explicitly acknowledges and accommodates disagreement rather than seeking to eliminate it.

\textbf{Tolerance vs. Acceptance**: Acceptance implies that one finds others' practices agreeable or appropriate. Tolerance explicitly allows for disapproval while still choosing non-interference. This distinction is crucial because it means tolerance can be achieved even when value consensus is impossible.

\textbf{Tolerance vs. Prejudice Reduction**: Traditional prejudice reduction seeks to eliminate negative affect toward outgroups. Tolerance accepts that some negative evaluations may be principled (e.g., disagreement with practices that conflict with gender equality) while still promoting behavioral norms of equal treatment.

\textbf{Tolerance vs. Equality-Based Respect**: \citet{simon2023respect} concept of equality-based respect provides the psychological foundation for tolerance by emphasizing individuals' equal citizenship rights rather than personal approval. Tolerance can be grounded in respect for democratic principles rather than personal affinity.

This conceptual precision makes tolerance potentially more achievable than alternatives in contexts of deep value differences, because it does not require individuals to abandon their principled convictions.

\subsection{Interethnic Cooperation}

Interethnic cooperation is defined as \textbf{voluntary collaborative behavior between individuals from different ethnic groups toward shared goals}. In the school context examined in this research, cooperation is operationalized through several behavioral indicators:

\begin{itemize}
\item \textbf{Academic Collaboration}: Working together on school projects and assignments
\item \textbf{Resource Sharing}: Lending materials and providing academic assistance
\item \textbf{Social Support}: Providing emotional or instrumental support during difficulties
\item \textbf{Joint Activities**: Participating together in extracurricular activities
\end{itemize}

Cooperation is measured at the dyadic level through network data, allowing examination of how tolerance affects the formation and maintenance of cooperative ties across ethnic boundaries. This dyadic focus enables several analytical advantages:

\textbf{Causal Identification}: Dyadic analysis enables identification of individual-level effects on cooperation formation while controlling for partner characteristics and dyadic factors.

\textbf{Ethnic Boundary Analysis**: The focus on cooperation tie formation allows direct examination of whether tolerance specifically promotes interethnic cooperation or simply increases cooperation generally.

\textbf{Behavioral Validation**: Using observed cooperation behavior rather than self-reported intentions provides stronger evidence for real-world tolerance effects.

\textbf{Network Co-evolution**: Dyadic analysis enables examination of how tolerance and cooperation co-evolve over time through reciprocal influence and selection processes.

\subsection{Friendship Networks}

Friendship networks provide the structural foundation through which social influence operates. Friendship is conceptualized as \textbf{affective ties characterized by mutual liking, trust, and voluntary association}. Research consistently demonstrates that friendship ties are the primary channels through which attitudes and behaviors spread in adolescent populations \citep{mercken2010dynamics}.

The theoretical framework assumes that tolerance attitudes spread primarily through close friendship networks rather than through broader peer networks, cooperation networks, or random contact. This assumption is based on extensive research showing that attitude change requires trust, perceived similarity, and credibility—characteristics that define close friendship relationships \citep{friedkin2011social}.

This friend-based influence assumption has several important implications:

\textbf{Network Structure Dependency**: The effectiveness of tolerance interventions depends critically on the friendship network structure within classrooms. Highly segregated friendship networks may limit tolerance diffusion across ethnic boundaries.

\textbf{Relationship Quality Effects**: The strength and quality of friendship relationships may moderate influence effectiveness, with stronger friendships enabling influence across larger attitude distances.

\textbf{Multiple Friend Effects**: Individuals with multiple tolerant friends should show greater tolerance change than those with single tolerant friends, consistent with complex contagion requiring social reinforcement.

\textbf{Intervention Targeting Implications**: Successful intervention targeting requires understanding both network structure (who is connected to whom) and relationship quality (how strong are these connections).

\section{Theoretical Mechanisms}

\subsection{Social Influence Mechanism: Social Judgment Theory Foundation}

The theoretical framework proposes that tolerance spreads through friendship networks via an \textbf{attraction-repulsion influence mechanism} that is grounded in Social Judgment Theory \citep{sherif1961social, sherif1965attitude}. This theoretical foundation provides both psychological realism and predictive precision for understanding tolerance diffusion processes.

\subsubsection{Social Judgment Theory Foundations}

Social Judgment Theory provides the core psychological mechanism underlying tolerance influence processes. The theory proposes that individuals evaluate persuasive messages (including friends' tolerance attitudes) relative to their own current attitude positions. Each individual possesses three latitudes around their current attitude:

\begin{enumerate}
\item \textbf{Latitude of Acceptance}: Positions that the individual finds acceptable or reasonable
\item \textbf{Latitude of Rejection**: Positions that the individual finds unacceptable or unreasonable
\item \textbf{Latitude of Non-commitment**: Positions about which the individual is uncertain or ambivalent
\end{enumerate}

Crucially, the theory predicts different influence outcomes depending on which latitude contains the friend's attitude position:

\textbf{Assimilation Effects**: When friends' tolerance levels fall within the latitude of acceptance, individuals will be attracted toward those positions, leading to attitude convergence.

\textbf{Contrast Effects**: When friends' tolerance levels fall within the latitude of rejection, individuals will be repelled from those positions, leading to attitude divergence.

\textbf{Variable Effects**: When friends' tolerance levels fall within the latitude of non-commitment, influence outcomes depend on other contextual factors such as relationship strength and message framing.

\subsubsection{Mathematical Formalization}

The attraction-repulsion influence mechanism is formalized as:

\begin{equation}
\Delta tolerance_i = f(|tolerance_i - tolerance_j|, \theta_{acceptance}, \theta_{rejection})
\end{equation}

Where:
\begin{align}
f(diff, \theta_{acceptance}, \theta_{rejection}) = \begin{cases}
0 & \text{if } diff < \theta_{min} \text{ (no new information)} \\
-\alpha \cdot diff & \text{if } \theta_{min} \leq diff \leq \theta_{acceptance} \text{ (assimilation)} \\
0 & \text{if } \theta_{acceptance} < diff < \theta_{rejection} \text{ (non-commitment)} \\
\beta \cdot diff & \text{if } diff \geq \theta_{rejection} \text{ (contrast)}
\end{cases}
\end{align}

This formalization captures four distinct influence regimes:

\textbf{No Influence Zone} ($diff < \theta_{min}$): When tolerance levels are very similar, no influence occurs because there is insufficient new information to motivate attitude change.

\textbf{Assimilation Zone} ($\theta_{min} \leq diff \leq \theta_{acceptance}$): When tolerance differences are moderate and fall within the latitude of acceptance, individuals move toward their friends' positions. The negative coefficient $\alpha$ indicates convergence.

\textbf{Non-commitment Zone} ($\theta_{acceptance} < diff < \theta_{rejection}$): When tolerance differences fall within the latitude of non-commitment, influence effects are uncertain and may depend on other factors such as relationship strength or repeated exposure.

\textbf{Contrast Zone} ($diff \geq \theta_{rejection}$): When tolerance differences are extreme and fall within the latitude of rejection, individuals move away from their friends' positions. The positive coefficient $\beta$ indicates divergence.

\subsubsection{Individual and Contextual Variation}

The parameters $\theta_{acceptance}$ and $\theta_{rejection}$ are not fixed but vary across individuals and contexts. Several factors influence the width of these latitudes:

\textbf{Ego Involvement**: Individuals with high ego involvement (strong personal investment in tolerance issues) have narrower latitudes of acceptance and wider latitudes of rejection, making them more resistant to influence.

\textbf{Attitude Extremity**: Individuals with extreme tolerance positions (very high or very low) have narrower latitudes of acceptance, making them less susceptible to influence from moderates.

\textbf{Relationship Strength**: Stronger friendship ties may expand the latitude of acceptance, allowing for influence across larger attitude distances.

\textbf{Group Context**: The tolerance climate of the broader peer group may influence individual latitudes, with more tolerant contexts expanding acceptance latitudes.

\subsubsection{Implications for Intervention Design}

The Social Judgment Theory foundation generates specific predictions for tolerance intervention effectiveness:

\textbf{Moderate Targeting Advantage**: Targeting individuals with moderate tolerance levels should be most effective because they have the largest latitudes of acceptance relative to the broader population.

\textbf{Extreme Position Risk**: Interventions that create extreme tolerance positions may be counterproductive by falling within others' latitudes of rejection.

\textbf{Gradual Change Strategy**: Sequential interventions that gradually shift tolerance norms may be more effective than single interventions that attempt large changes.

\textbf{Context Preparation**: Interventions may need to first expand latitudes of acceptance (through relationship building or climate change) before attempting attitude change.

\subsection{Selection Mechanism: Trust Expansion and Radius of Trust}

The theoretical framework proposes that tolerance affects interethnic cooperation through a \textbf{trust expansion mechanism} that operates by expanding individuals' "radius of trust" beyond their immediate ethnic ingroup. This mechanism provides the crucial theoretical link between tolerance attitudes and cooperative behaviors.

\subsubsection{Theoretical Foundations}

The trust expansion mechanism draws on several theoretical traditions:

\textbf{Social Identity Theory}: Individuals naturally favor their ethnic ingroup and may be suspicious of outgroup members \citep{tajfel1979integrative}. Tolerance attitudes can override these natural tendencies by promoting recognition of outgroup members' equal citizenship rights.

\textbf{Intergroup Contact Theory**: Positive contact experiences can reduce intergroup anxiety and increase trust \citep{pettigrew1998intergroup}. However, tolerance can promote cooperation even in the absence of prior positive contact by establishing behavioral norms of equal treatment.

\textbf{Equality-Based Respect}: \citet{simon2023respect} concept of equality-based respect suggests that individuals can cooperate with dissimilar others based on recognition of equal citizenship rights rather than personal affinity. This provides a cognitive pathway to cooperation that does not require affective change.

\subsubsection{Radius of Trust Concept}

The "radius of trust" represents the social distance beyond which individuals are unwilling to engage in cooperative relationships. For most individuals, this radius is naturally limited to their ethnic ingroup due to perceived similarity, shared identity, and reduced uncertainty.

Tolerance attitudes expand this radius by:

\begin{enumerate}
\item \textbf{Reducing Perceived Risk**: Tolerant individuals perceive lower social and personal risk from interacting with ethnically different others
\item \textbf{Increasing Legitimacy Recognition**: Tolerance involves recognizing others' legitimate rights to different practices and equal treatment
\item \textbf{Overriding Group Bias**: Tolerance provides principled reasons for cooperation that can override natural ingroup preferences
\item \textbf{Establishing Behavioral Norms**: Tolerance creates behavioral expectations of non-discrimination and equal treatment
\end{enumerate}

\subsubsection{Mathematical Formalization}

The trust expansion mechanism is formalized as:

\begin{equation}
P(cooperation_{ij} = 1) = \text{logit}^{-1}(\beta_0 + \beta_1 \cdot tolerance_i + \beta_2 \cdot ethnicity\_diff_{ij} + \beta_3 \cdot tolerance_i \times ethnicity\_diff_{ij})
\end{equation}

Where:
\begin{itemize}
\item $P(cooperation_{ij} = 1)$ is the probability that individual $i$ initiates cooperation with individual $j$
\item $tolerance_i$ is individual $i$'s tolerance level (standardized 0-1 scale)
\item $ethnicity\_diff_{ij}$ is an indicator for whether individuals $i$ and $j$ are from different ethnic groups
\item The interaction term $tolerance_i \times ethnicity\_diff_{ij}$ captures how tolerance moderates the effect of ethnic differences on cooperation
\end{itemize}

The theoretical predictions are:

\begin{itemize}
\item $\beta_1 > 0$: Higher tolerance increases overall cooperation propensity
\item $\beta_2 < 0$: Ethnic differences generally reduce cooperation probability (ingroup bias)
\item $\beta_3 > 0$: Tolerance reduces the negative effect of ethnic differences on cooperation (radius expansion)
\end{itemize}

The key theoretical prediction is that $\beta_3 > 0$, indicating that tolerance specifically promotes interethnic cooperation by reducing the cooperation penalty associated with ethnic differences.

\subsubsection{Boundary Conditions and Moderators}

The effectiveness of the trust expansion mechanism may vary based on several boundary conditions:

\textbf{Domain Specificity**: Tolerance may promote cooperation in some domains (academic collaboration) more than others (intimate friendship formation).

\textbf{Institutional Context**: The mechanism may be stronger in institutional contexts (schools) that emphasize equal treatment norms compared to informal contexts.

\textbf{Ethnic Distance**: The mechanism may be more effective for promoting cooperation between groups with moderate cultural differences compared to groups with extreme differences.

\textbf{Individual Characteristics**: The mechanism may be stronger for individuals with higher cognitive sophistication or democratic values orientation.

\subsection{Network Structure Effects}

The theoretical framework proposes that network structures moderate both influence and selection processes:

\textbf{Clustering Effects}: Dense local networks (high clustering) facilitate complex contagion by providing multiple sources of influence and social reinforcement for tolerance norms. However, high clustering may also impede tolerance spread by creating echo chambers that resist external influence.

\textbf{Bridging Effects}: Individuals who bridge different ethnic groups (high betweenness centrality) play crucial roles in tolerance diffusion by connecting otherwise separated communities. However, these bridge individuals may face conflicting social pressures that inhibit their own tolerance development.

\textbf{Popularity Effects**: Highly popular individuals (high degree centrality) can accelerate tolerance spread when they adopt tolerant attitudes, but they may also face pressure to conform to existing group norms.

\section{Empirical Grounding: The German High School Context}

The theoretical framework is empirically grounded in the German high school context studied by \citet{shani2023tolerance}, which provides both substantive foundation and methodological validation for the theoretical mechanisms.

\subsection{The Shani et al. (2023) Study}

\textbf{Study Context}: The research was conducted in diverse German high schools with substantial Turkish and Middle Eastern immigrant populations, representing a natural laboratory for studying ethnic majority-minority relations in European educational settings.

\textbf{Intervention Design**: The tolerance intervention targeted ethnic majority (German) students and emphasized equality-based respect: "even when we disagree with others' practices, we should respect their equal rights as citizens." This approach explicitly avoided prejudice reduction appeals and instead focused on behavioral norms of tolerance.

\textbf{Targeting Strategies}: The study compared random selection with popularity-based targeting, providing direct empirical evidence about the relative effectiveness of different intervention approaches.

\textbf{Key Empirical Insights}:
\begin{enumerate}
\item \textbf{Limited Overall Effectiveness**: The intervention showed modest effects, highlighting the difficulty of changing tolerance attitudes and the need for more sophisticated theoretical understanding
\item \textbf{Network Structure Dependence}: Intervention effectiveness varied significantly across classrooms with different friendship network structures
\item \textbf{Popularity Targeting Advantage**: Popular student targeting showed modest but consistent advantages over random targeting
\item \textbf{Baseline Heterogeneity}: Treatment effects varied substantially based on students' initial tolerance levels and network positions
\end{enumerate}

\subsection{Theoretical Implications}

The German context provides crucial theoretical insights that inform the current framework:

\textbf{Complex Contagion Evidence**: The limited diffusion suggests tolerance change requires multiple exposures and social reinforcement rather than simple exposure.

\textbf{Social Judgment Mechanisms**: Variation in treatment effects by baseline tolerance levels is consistent with Social Judgment Theory predictions about latitudes of acceptance and rejection.

\textbf{Popular vs. Central Actor Distinction**: The modest advantage of popularity-based targeting supports the theoretical emphasis on credibility over mere connectivity.

\textbf{Majority Group Focus}: Unlike most research focusing on minority group attitudes, the German context examines majority group tolerance, reflecting practical policy concerns about majority power to implement discrimination.

\section{Formal Hypotheses and Testable Predictions}

Based on the Social Judgment Theory foundation and trust expansion mechanism, this research tests six core hypotheses that operationalize the theoretical framework.

\subsection{Social Influence Hypotheses}

\begin{hypothesis}[Social Judgment Influence Hypothesis]
\textbf{H1}: Tolerance change will follow Social Judgment Theory predictions, with assimilation effects (convergence) occurring when friendship partners have moderate tolerance differences within the latitude of acceptance, and contrast effects (divergence) occurring when differences are extreme and fall within the latitude of rejection.

\textbf{Operationalization}: Tolerance change rates will be positive (convergence) for moderate friendship tolerance differences (|difference| $\leq$ threshold) and negative (divergence) for extreme differences (|difference| > threshold), with the threshold estimated empirically.
\end{hypothesis}

\begin{hypothesis}[Friend-Based Influence Hypothesis]
\textbf{H2}: Tolerance influence will operate primarily through close friendship ties rather than broader classroom-level exposure, consistent with the trust and credibility requirements of attitude change processes.

\textbf{Operationalization**: Friend tolerance levels will predict individual tolerance change while controlling for classroom-level tolerance climate and other peer exposures.
\end{hypothesis}

\subsection{Selection and Cooperation Hypotheses}

\begin{hypothesis}[Trust Expansion Hypothesis]
\textbf{H3**: Higher tolerance will increase willingness to form cooperative relationships with ethnically different others, demonstrating the radius of trust expansion mechanism.

\textbf{Operationalization**: The interaction effect $\beta_3$ in the cooperation formation equation will be significantly positive, indicating that tolerance reduces the cooperation penalty associated with ethnic differences.
\end{hypothesis}

\begin{hypothesis}[Behavioral Specificity Hypothesis]
\textbf{H4**: Tolerance effects on cooperation will be stronger for instrumental cooperation (academic collaboration) than for affective relationships (close friendship), reflecting domain-specific trust expansion.

\textbf{Operationalization**: The tolerance × ethnic difference interaction will be larger for academic cooperation networks than for friendship networks.
\end{hypothesis}

\subsection{Intervention Design Hypotheses}

\begin{hypothesis}[Complex Contagion Hypothesis]
\textbf{H5**: Tolerance interventions will be more effective when delivered to clustered groups of connected individuals rather than isolated individuals, consistent with complex contagion dynamics requiring social reinforcement.

\textbf{Operationalization**: Simulated interventions targeting clustered individuals will produce greater tolerance change and cooperation increases than interventions targeting randomly selected individuals with equivalent network centrality.
\end{hypothesis}

\begin{hypothesis}[Popular Actor Targeting Hypothesis]
\textbf{H6**: Tolerance interventions targeting popular actors (high peer respect) will produce greater diffusion effects than interventions targeting structurally central actors (high degree centrality), reflecting the importance of credibility for norm change.

\textbf{Operationalization**: Simulated interventions targeting individuals with high popularity ratings will produce greater network-wide tolerance change than interventions targeting individuals with equivalent structural centrality but lower popularity.
\end{hypothesis}

\subsection{Secondary Hypotheses}

\begin{hypothesis}[Dosage Effects Hypothesis]
\textbf{H7**: Moderate tolerance increases will produce greater network diffusion than extreme increases, consistent with Social Judgment Theory predictions about latitude of rejection effects.

\textbf{Operationalization**: Simulated interventions with moderate tolerance increases will produce greater network-wide change than interventions with extreme increases of equivalent magnitude.
\end{hypothesis}

\begin{hypothesis}[Feedback Loop Hypothesis]
\textbf{H8**: Tolerance intervention effects will be amplified over time through positive feedback loops between tolerance attitudes and cooperative behaviors.

\textbf{Operationalization**: Longitudinal simulations will show increasing intervention effects over time as tolerance changes promote cooperation formation, which in turn reinforces tolerance attitudes.
\end{hypothesis}

\section{Scope Conditions and Assumptions}

The theoretical framework operates under several scope conditions and assumptions:

\textbf{Adolescent Population**: The theory is developed specifically for adolescent populations in school settings, where friendship networks are particularly important for social influence. Generalization to other age groups and contexts requires additional research.

\textbf{Moderate Ethnic Differences**: The theory assumes contexts where ethnic differences are salient but not extreme. In contexts of intense ethnic conflict or extreme prejudice, different mechanisms may operate.

\textbf{Voluntary Association**: The theory assumes that cooperation occurs through voluntary choice rather than institutional mandate. Forced cooperation may not reflect genuine tolerance attitudes.

\textbf{Observable Networks}: The theory requires that friendship and cooperation networks are observable and relatively stable over the time periods examined.

\textbf{Individual Agency**: The theory assumes that individuals have meaningful choice in their friendship and cooperation decisions. Structural constraints (e.g., segregated housing) may limit the applicability of tolerance interventions.

\section{Theoretical Integration and Novel Contributions}

The proposed framework integrates several established theories while extending them in theoretically and methodologically significant ways.

\subsection{Integration with Established Theories}

\textbf{Social Judgment Theory Extension}: While Social Judgment Theory has been applied primarily to attitude change in laboratory settings, this framework extends it to network-based influence processes in naturalistic settings. The mathematical formalization of latitudes provides precise predictions for influence outcomes in social networks.

\textbf{Contact Theory Reconceptualization**: Contact theory focuses on attitude change through positive intergroup experiences. The current framework reconceptualizes intergroup relations by focusing on behavioral accommodation that can occur even without attitude change or positive contact experiences. This represents a fundamental shift from affective to behavioral mechanisms.

\textbf{Social Identity Theory Accommodation**: Social identity theory predicts ingroup bias and outgroup derogation as natural psychological tendencies. Rather than attempting to eliminate these tendencies (as in traditional prejudice reduction), the tolerance framework accommodates them while promoting behavioral norms of equal treatment.

\textbf{Complex Contagion Specification**: Complex contagion theory provides general principles about social influence requiring multiple exposures. This framework specifies these principles for tolerance attitudes, identifying the particular conditions (clustered targeting, moderate changes) under which complex contagion operates for attitude change.

\textbf{Network Co-evolution Theory**: Network theories typically examine influence and selection processes separately or assume they operate independently. This framework models their dynamic interaction, showing how tolerance interventions trigger cascading changes in both attitudes and network structure.

\subsection{Novel Theoretical Contributions}

\textbf{Tolerance-Cooperation Mechanism**: The radius of trust expansion mechanism provides the first formal theoretical link between tolerance attitudes and cooperative behaviors. This mechanism explains how attitude interventions can produce behavioral change without requiring affective change.

\textbf{Social Judgment Network Integration**: The integration of Social Judgment Theory with network influence processes provides a psychologically realistic foundation for network intervention design. This integration addresses the "black box" problem in network research by specifying precise psychological mechanisms.

\textbf{Popular Actor vs. Central Actor Distinction**: The theoretical distinction between structural centrality (network position) and social popularity (peer respect) provides a novel approach to intervention targeting that emphasizes credibility over connectivity.

\textbf{Intervention Design Theory**: The framework provides a comprehensive theory of intervention design that considers targeting strategies, delivery approaches, and dosage effects within a unified framework based on complex contagion and Social Judgment Theory principles.

\textbf{Feedback Loop Specification**: The theoretical specification of positive feedback loops between tolerance attitudes and cooperative behaviors provides a mechanism for understanding how short-term interventions can produce long-term change through self-reinforcing processes.

\subsection{Methodological Contributions}

\textbf{ABM-SAOM Integration}: The framework provides theoretical justification for integrating agent-based modeling (for intervention simulation) with stochastic actor-oriented modeling (for empirical validation). This integration addresses key limitations of both approaches.

\textbf{Behavioral Outcome Focus**: By focusing on cooperative behaviors rather than attitudes alone, the framework addresses the attitude-behavior gap that has plagued intergroup relations research.

\textbf{Predictive Intervention Modeling**: The framework enables predictive modeling of intervention effects before implementation, moving beyond post-hoc evaluation toward optimization-based intervention design.

\section{Empirical Validation Strategy}

The theoretical framework generates specific testable predictions that guide both observational analysis and simulation experiments.

\subsection{Observational Analysis Predictions}

Based on the theoretical mechanisms, the framework generates the following predictions for observational data analysis:

\begin{enumerate}
\item \textbf{Non-linear Influence Pattern**: Tolerance change rates will follow a non-monotonic pattern with respect to friend tolerance differences, with maximum change at moderate differences and minimal change at both small and extreme differences

\item \textbf{Cooperation Selection Pattern**: The probability of cooperation tie formation will increase with ego tolerance levels, with stronger effects for interethnic dyads than same-ethnic dyads

\item \textbf{Friend-Based Influence Specificity**: Friend tolerance levels will predict tolerance change while classroom-level tolerance climate will have minimal effects after controlling for friend influences

\item \textbf{Behavioral Domain Differences**: Tolerance effects will be stronger for instrumental cooperation (academic collaboration) than for affective relationships (close friendship)

\item \textbf{Network Position Moderation**: Individual network positions will moderate both influence susceptibility and influence effectiveness, with central and popular individuals showing different patterns
\end{enumerate}

\subsection{Simulation Experiment Predictions}

The framework generates specific predictions for agent-based simulation experiments testing intervention strategies:

\begin{enumerate}
\item \textbf{Targeting Strategy Effects**: Popular actor targeting will produce greater network-wide tolerance change than random targeting or structural centrality targeting

\item \textbf{Delivery Strategy Effects**: Clustered delivery (targeting connected groups) will produce greater tolerance change than dispersed delivery (targeting isolated individuals)

\item \textbf{Dosage Effects**: Moderate tolerance increases will produce greater network diffusion than extreme increases of equivalent magnitude

\item \textbf{Temporal Dynamics}: Intervention effects will show non-linear temporal patterns, with initial resistance followed by accelerating change as social reinforcement accumulates

\item \textbf{Feedback Amplification**: Long-term simulations will show amplification of intervention effects through positive feedback between tolerance and cooperation
\end{enumerate}

\subsection{Integration and Validation}

The empirical validation strategy integrates observational and simulation analyses through a two-stage process:

\textbf{Stage 1: Observational Validation}: SAOM analysis of longitudinal network data tests the core theoretical mechanisms (Social Judgment influence, trust expansion selection) and estimates key parameters.

\textbf{Stage 2: Simulation Application**: ABM simulations use empirically validated parameters to test intervention strategies and generate predictions for intervention effectiveness under different conditions.

This integration provides both empirical grounding for the theoretical mechanisms and practical guidance for intervention design optimization.

\section{Conclusion}

\section{Conclusion: Toward a Comprehensive Theory of Tolerance Intervention Effectiveness}

This theoretical framework represents a significant advance in understanding how tolerance interventions can promote interethnic cooperation through social influence processes in networks. The framework makes several key theoretical and methodological contributions:

\subsection{Theoretical Contributions}

\textbf{Psychological Realism**: By grounding network influence processes in Social Judgment Theory, the framework provides psychologically realistic mechanisms that explain when and why influence occurs.

\textbf{Behavioral Focus}: By emphasizing the tolerance-cooperation pathway through trust expansion mechanisms, the framework addresses the attitude-behavior gap that has limited previous intergroup intervention research.

\textbf{Network Integration**: By specifying how individual psychological processes aggregate to produce network-level patterns, the framework bridges micro-level psychology and macro-level social structure.

\textbf{Intervention Optimization**: By providing specific predictions about targeting strategies, delivery approaches, and dosage effects, the framework enables evidence-based intervention design.

\subsection{Methodological Innovations}

The framework enables several methodological innovations that advance computational social science methodology:

\textbf{Theory-Driven ABM}: The Social Judgment Theory foundation provides theoretical justification for agent behavioral rules, addressing the "ad hoc" criticism of agent-based modeling.

\textbf{Empirical Parameter Validation**: The SAOM integration enables empirical estimation of key theoretical parameters, grounding simulations in real social processes.

\textbf{Predictive Intervention Modeling**: The combination of empirical validation and simulation experimentation enables predictive modeling of intervention effects before implementation.

\subsection{Policy Relevance}

Beyond its theoretical and methodological contributions, the framework has direct policy relevance for educational institutions and diversity programs:

\textbf{Evidence-Based Design**: The framework provides specific guidance for designing tolerance interventions based on scientific evidence about social influence processes.

\textbf{Cost-Effectiveness Optimization**: By identifying optimal targeting and delivery strategies, the framework enables more efficient use of limited intervention resources.

\textbf{Realistic Expectations**: By specifying the conditions under which interventions succeed or fail, the framework enables more realistic expectations about intervention effectiveness.

The next chapter operationalizes this theoretical framework through a comprehensive methodological approach that integrates agent-based modeling with stochastic actor-oriented modeling to test theoretical predictions and optimize intervention strategies.