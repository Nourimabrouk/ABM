\chapter{Literature Review}

\section{Introduction}

This chapter reviews the theoretical and empirical literature that informs our understanding of tolerance, interethnic cooperation, and social influence in networks. The review is organized around four key domains: (1) traditional approaches to prejudice reduction and intergroup contact, (2) emerging research on tolerance as an alternative to prejudice reduction, (3) theories of social influence and attitude change in networks, and (4) intervention strategies in social networks. By synthesizing these literatures, this chapter identifies theoretical gaps and establishes the foundation for the current research.

\section{Intergroup Contact and Prejudice Reduction}

\subsection{The Contact Hypothesis}

The contact hypothesis, originally formulated by \citet{allport1954nature}, has dominated research on improving intergroup relations for over half a century. Allport proposed that prejudice between groups could be reduced through interpersonal contact under four optimal conditions: equal status between groups, common goals, intergroup cooperation, and support from authorities. This simple yet powerful idea has generated thousands of studies and numerous meta-analyses examining when and how contact reduces prejudice.

Early research provided strong support for the contact hypothesis under laboratory conditions and in structured interventions. \citet{sherif1961robbers} classic Robbers Cave experiment demonstrated that superordinate goals requiring intergroup cooperation could reduce hostility between groups of boys at summer camp. Similarly, \citet{aronson1978jigsaw} showed that cooperative learning structures in diverse classrooms could improve intergroup attitudes and academic outcomes.

\subsection{Meta-Analytic Evidence}

Meta-analyses have generally supported the effectiveness of intergroup contact for reducing prejudice. \citet{pettigrew1998intergroup} comprehensive analysis of 515 studies found a significant negative relationship between contact and prejudice (r = -.21), with larger effects when Allport's optimal conditions were met. \citet{pettigrew2006meta} updated analysis with 696 samples confirmed these findings and demonstrated that contact effects generalize beyond the immediate outgroup to other outgroups not involved in the contact situation.

However, more recent meta-analyses have raised important questions about the practical significance and limitations of contact interventions. \citet{paluck2019contact} analysis of 27 studies specifically examining the effects of intergroup contact on discrimination and collective action found much smaller effects (d = .16) than studies measuring attitudinal outcomes. Crucially, many studies showed positive effects on self-reported attitudes but no significant effects on actual behavior toward outgroup members.

\subsection{Limitations of Contact Approaches}

Several limitations of traditional contact approaches have become apparent through decades of research and implementation:

\textbf{Implementation Challenges}: Allport's optimal conditions are difficult to achieve in real-world settings. Natural contact often occurs under suboptimal conditions with status differences, competition rather than cooperation, and limited institutional support \citep{dixon2005beyond}.

\textbf{Self-Selection}: Individuals who seek intergroup contact may already hold more positive attitudes toward outgroups, creating selection effects that limit causal inference about contact effects \citep{binder2009does}.

\textbf{Attitude-Behavior Gap}: Even when contact successfully reduces explicit prejudice, this does not guarantee changes in discriminatory behavior or support for policies that benefit outgroups \citep{tropp2008attitude}.

\textbf{Majority Group Focus}: Contact interventions often focus primarily on changing majority group attitudes while placing the burden of contact on minority group members, potentially reproducing rather than challenging existing power dynamics \citep{dixon2012what}.

\textbf{Persistence}: Effects of contact interventions often decay over time, particularly when individuals return to segregated environments that do not reinforce positive intergroup experiences \citep{lemmer2015can}.

\section{Tolerance as Alternative to Prejudice Reduction}

\subsection{Conceptualizing Tolerance}

In response to limitations of prejudice-focused approaches, scholars have increasingly turned attention to tolerance as an alternative pathway to intergroup harmony. Tolerance is typically defined as "the willingness to allow or accept something that one dislikes or disagrees with" \citep{verkuyten2020tolerance}. This definition highlights tolerance as a behavioral choice rather than an attitudinal state—one can disapprove of others' practices while still choosing to tolerate them.

\citet{verkuyten2020tolerance} distinguishes between tolerance and related concepts such as acceptance and respect. While acceptance implies agreement or positive evaluation, tolerance explicitly involves disagreement or dislike while still choosing non-interference. This distinction is crucial because it suggests that tolerance may be achievable even when attitude change is impossible or undesirable.

\subsection{Philosophical Foundations}

Tolerance has deep philosophical roots in liberal political theory. \citet{rawls1993political} concept of "overlapping consensus" suggests that citizens with different comprehensive doctrines can agree on political principles without sharing underlying beliefs. Similarly, \citet{habermas1996between} emphasizes tolerance as a civic virtue necessary for democratic participation in diverse societies.

\citet{simon2023respect} recent work on "equality-based respect" provides a foundation for tolerance that does not require agreement on values. According to this perspective, individuals can respect others as equal citizens with legitimate rights to hold different views while still maintaining their own principled disagreements. This approach shifts focus from changing minds to establishing behavioral norms of mutual respect.

\subsection{Tolerance versus Prejudice Reduction: A Fundamental Theoretical Distinction}

The distinction between tolerance and prejudice reduction represents a fundamental theoretical advancement in intergroup relations research. This distinction is not merely semantic but reflects different assumptions about human psychology, social change mechanisms, and intervention goals \citep{verkuyten2020tolerance, simon2023respect}.

\textbf{Definitional Precision}: Tolerance is defined as the behavioral intention to refrain from interfering with others' practices or expressions despite personal disapproval or disagreement. This definition contains three critical components: (1) continued disagreement or disapproval (cognitive component), (2) behavioral restraint from interference (behavioral component), and (3) recognition of others' legitimate rights to different practices (normative component). Prejudice reduction, by contrast, seeks to eliminate or minimize negative affect, stereotyping, and discriminatory intentions toward outgroup members.

\textbf{Psychological Mechanisms}: The psychological pathways to tolerance and prejudice reduction operate through different mechanisms. Prejudice reduction relies heavily on affective change—reducing negative emotions, increasing empathy, and fostering positive intergroup contact \citep{pettigrew1998intergroup}. Tolerance, however, can be achieved through equality-based respect mechanisms that do not require affective change \citep{simon2023respect}. Individuals can maintain their principled disagreements while simultaneously recognizing others' equal citizenship rights and democratic legitimacy.

\textbf{Cognitive versus Behavioral Focus}: Prejudice reduction focuses primarily on changing internal cognitive and affective states, with the assumption that behavioral change will follow from attitude change. Tolerance approaches focus directly on behavioral intentions and choices, accepting that internal attitudes may remain unchanged. This distinction has profound implications for intervention design and measurement.

\textbf{Disagreement Legitimacy}: Perhaps most fundamentally, prejudice reduction approaches typically assume that intergroup disagreements reflect ignorance, bias, or malice that should be eliminated. Tolerance approaches accept that many intergroup disagreements reflect legitimate value differences that should be respected rather than eliminated \citep{rawls1993political}.

\textbf{Temporal Sustainability}: Recent research suggests that tolerance-based changes may be more sustainable than prejudice reduction because they do not require ongoing attitude maintenance\citep{verkuyten2023tolerance}. Once individuals develop behavioral norms of non-interference, these can persist even when affective attitudes fluctuate.

\subsection{Empirical Research on Tolerance}

Empirical research on tolerance interventions is still emerging, but early studies show promise. \citet{verkuyten2023tolerance} experimental studies demonstrate that tolerance-based appeals can be more effective than prejudice-reduction approaches for promoting support for minority rights among individuals with strong religious or cultural convictions.

\citet{simon2023respect} field experiments in diverse neighborhoods found that interventions emphasizing equality-based respect increased residents' willingness to engage in cooperative behaviors with dissimilar neighbors, even when attitudinal measures showed no change.

However, tolerance research faces several challenges. First, the behavioral nature of tolerance makes it more difficult to measure than attitudes. Second, tolerance may be context-specific, making it unclear how tolerance in one domain transfers to others. Third, the relationship between tolerance and long-term intergroup harmony remains underexplored.

\section{Social Influence in Networks}

\subsection{Classical Theories of Social Influence}

Understanding how tolerance spreads through social networks requires examining classical theories of social influence. \citet{sherif1961social} social judgment theory proposes that individuals evaluate new information relative to their existing attitudes. Messages falling within one's "latitude of acceptance" lead to attitude change in the direction of the message, while messages in the "latitude of rejection" may produce boomerang effects, causing attitudes to move away from the message.

\citet{festinger1957theory} cognitive dissonance theory suggests that individuals are motivated to maintain consistency between their attitudes and behaviors. When inconsistency arises, individuals may change either attitudes or behaviors to restore consonance. This process can be triggered by observing others' attitudes and behaviors that differ from one's own.

Social comparison theory \citep{festinger1954theory} emphasizes that individuals evaluate their own attitudes and abilities relative to similar others. This creates pressure for attitude convergence within social groups, as individuals adjust their positions to maintain group membership and reduce uncertainty.

\subsection{Network Theories of Social Influence}

Network approaches to social influence recognize that influence occurs through social connections rather than random encounters. Several key mechanisms have been identified:

\textbf{Structural Cohesion}: \citet{coleman1988social} argued that dense networks create social capital that facilitates information transmission and norm enforcement. In cohesive networks, deviant attitudes or behaviors are quickly identified and corrected through social pressure.

\textbf{Structural Equivalence}: \citet{burt1987social} structural equivalence theory suggests that individuals in similar network positions face similar social pressures and therefore develop similar attitudes and behaviors, even without direct contact.

\textbf{Centrality Effects}: Research on opinion leadership \citep{katz1957personal} and network centrality \citep{freeman1978centrality} suggests that well-connected individuals have disproportionate influence on others' attitudes and behaviors.

\subsection{Attraction-Repulsion Dynamics and Social Judgment Theory}

The attraction-repulsion dynamics observed in social influence research provide crucial theoretical foundation for understanding tolerance diffusion processes. This theoretical framework draws primarily on Social Judgment Theory \citep{sherif1961social}, which provides a precise psychological account of when and how attitude change occurs.

\subsubsection{Social Judgment Theory Foundations}

Social Judgment Theory proposes that individuals evaluate persuasive messages relative to their existing attitude positions. Each individual possesses three latitudes around their current attitude: (1) the latitude of acceptance, containing positions they find acceptable; (2) the latitude of rejection, containing positions they find unacceptable; and (3) the latitude of non-commitment, containing positions about which they are uncertain or ambivalent \citep{sherif1965attitude}.

Crucially, the theory predicts that messages falling within the latitude of acceptance will produce attitude change in the direction of the message (assimilation effects), while messages falling within the latitude of rejection will produce attitude change away from the message (contrast effects). Messages in the latitude of non-commitment may produce variable effects depending on other contextual factors.

\subsubsection{Application to Tolerance Diffusion}

When applied to tolerance diffusion, Social Judgment Theory suggests that individuals will be influenced by friends' tolerance levels only when those levels fall within moderate distances from their own positions. This creates the attraction-repulsion pattern observed in empirical research \citep{flache2017models, baldassarri2007dynamics}.

The theoretical prediction is formalized as:
\begin{equation}
P(influence) = f(|tolerance_i - tolerance_j|, \theta_{acceptance}, \theta_{rejection})
\end{equation}

Where influence probability follows a non-monotonic function of attitude distance, with maximum influence occurring at moderate distances and minimal influence at both very small and very large distances. This "Goldilocks principle" has profound implications for tolerance intervention design, suggesting that targeting individuals with extreme tolerance positions may be counterproductive if they fall within others' latitudes of rejection.

\subsubsection{Empirical Evidence}

Empirical research provides substantial support for attraction-repulsion dynamics in attitude change processes. \citet{baldassarri2007dynamics} longitudinal analysis of political attitudes found that moderate differences between discussion partners led to convergence, while extreme differences led to further polarization. \citet{flache2017models} agent-based models demonstrate how these individual-level processes can generate macro-level patterns of attitude clustering and polarization.

Recent experimental research by \citet{bail2018exposure} found that exposure to opposing political views on social media actually increased political polarization rather than promoting moderation, consistent with Social Judgment Theory predictions about contrast effects for extreme positions.

\subsection{Complex Contagion and Tolerance Diffusion Mechanisms}

The distinction between simple and complex contagion represents a crucial theoretical insight for understanding tolerance intervention effectiveness. This distinction has direct implications for optimal intervention design strategies.

\subsubsection{Theoretical Foundations}

Simple contagion models assume that exposure to a single adopter is sufficient for attitude or behavior change, following epidemiological models of disease transmission. Complex contagion models, by contrast, assume that multiple exposures and social reinforcement are required for adoption, particularly for behaviors that involve social risk or conflict with existing norms \citep{centola2010spread}.

Tolerance attitudes likely follow complex contagion patterns for several reasons: (1) they often conflict with existing prejudicial attitudes and social norms; (2) they require overcoming emotional responses and cognitive biases; (3) they involve social identity considerations that require social validation; and (4) they may involve personal risk when expressing tolerance in intolerant environments.

\subsubsection{Network Structure Implications}

The complex contagion assumption has profound implications for optimal network intervention strategies. \citet{centola2010spread} experimental research demonstrates that complex behaviors spread more effectively through clustered networks with redundant ties, while simple behaviors spread more effectively through bridging networks with non-redundant ties.

For tolerance interventions, this suggests that \textbf{clustered delivery strategies} that target groups of interconnected individuals may be more effective than \textbf{random delivery strategies} that target isolated individuals. Clustered delivery creates multiple sources of social reinforcement that can overcome resistance to tolerance change.

\subsubsection{Empirical Evidence}

Recent experimental research provides support for complex contagion mechanisms in attitude change. \citet{centola2018experimental} health behavior interventions found that clustered targeting was more effective than targeting central individuals for behaviors requiring social reinforcement. \citet{guilbeault2021experimental} coordination experiments demonstrated that complex contagion effects were strongest when multiple connected individuals were exposed simultaneously to new behaviors.

\section{Network Interventions}

\subsection{Targeting Strategies}

Network intervention research has identified several strategies for selecting intervention targets:

\textbf{High-Degree Targeting}: Selecting individuals with many connections maximizes the number of people exposed to intervention effects. This approach is based on epidemiological models of disease spread \citep{pastor2015epidemic}.

\textbf{High-Betweenness Targeting}: Selecting individuals who connect otherwise disconnected groups maximizes the spread of intervention effects across network boundaries \citep{valente2010network}.

\textbf{High-Eigenvector Targeting}: Selecting individuals who are connected to other well-connected individuals maximizes intervention effects through secondary influence \citep{bonacich2007some}.

\textbf{Peripheral Targeting}: Some research suggests that targeting peripheral individuals may be more effective because they face less social pressure to conform to existing norms \citep{centola2018experimental}.

\textbf{Popular Actor Targeting}: Recent research distinguishes between network centrality and social popularity, with evidence suggesting that popular individuals (those who are well-liked and respected) may be more effective intervention targets than simply well-connected individuals\citep{paluck2011peer}.

\textbf{Norm Entrepreneur Targeting}: Alternatively, targeting individuals who are willing to violate existing norms in favor of new ones (norm entrepreneurs) may be more effective for tolerance interventions that challenge existing prejudicial norms\citep{sunstein1996social}.

\subsection{Delivery Strategies}

Beyond targeting, network interventions must consider how to deliver intervention content:

\textbf{Seed and Spread}: Deliver intervention to select individuals who then influence others through natural social influence processes\citep{kim2015social}.

\textbf{Clustered Delivery}: Deliver intervention to groups of connected individuals to create local pockets of change that can resist social pressure\citep{centola2018experimental}.

\textbf{Sequential Delivery}: Deliver intervention in waves, allowing early adopters to influence subsequent targets\citep{bakshy2012role}.

\textbf{Coordinated Delivery}: Deliver intervention simultaneously across multiple network locations to create widespread momentum for change\citep{guilbeault2021experimental}.

\subsection{Empirical Evidence}

Empirical studies of network interventions have produced mixed results. \citet{kim2015social} study of smoking cessation found that targeting central individuals increased intervention reach but not necessarily effectiveness. \citet{centola2018experimental} study of health behavior change found that clustered delivery was more effective than targeting central individuals for complex behaviors requiring social reinforcement.

\citet{tankard2016norm} research on norm-based interventions found that highlighting the behaviors of "social referents"—respected community members—was more effective than highlighting behaviors of statistical majorities. This suggests that intervention effectiveness depends not just on network position but also on social status and credibility.

\section{Theoretical Integration and Research Implications}

\subsection{Synthesis of Theoretical Frameworks}

The integration of these diverse theoretical perspectives yields a comprehensive framework for understanding tolerance intervention effectiveness. This framework combines:

\begin{enumerate}
\item \textbf{Social Judgment Theory foundations}: Attitude change occurs through attraction-repulsion mechanisms based on perceived attitude distances
\item \textbf{Complex contagion dynamics}: Tolerance change requires multiple exposures and social reinforcement
\item \textbf{Equality-based respect mechanisms}: Tolerance can be achieved without requiring affective attitude change
\item \textbf{Network structure effects}: Intervention effectiveness depends on targeting strategies and network clustering patterns
\item \textbf{Tolerance-cooperation pathways}: Tolerance attitudes affect behavioral intentions toward intergroup cooperation
\end{enumerate}

\subsection{Intervention Design Implications}

The theoretical synthesis generates specific predictions about optimal intervention design:

\textbf{Targeting Strategy}: Popular actors who are respected by their peers may be more effective targets than simply well-connected individuals, because popularity provides credibility for norm change.

\textbf{Delivery Strategy}: Clustered delivery to groups of connected individuals should be more effective than random delivery to isolated individuals, because complex contagion requires social reinforcement.

\textbf{Dosage Considerations}: Moderate tolerance increases should be more effective than extreme increases, because extreme positions may fall within others' latitudes of rejection.

\textbf{Context Dependence}: Intervention effectiveness should depend on the existing tolerance climate, with greater effectiveness in contexts where tolerance levels are already moderate rather than extremely low.

\subsection{Research Gaps and Dissertation Contributions}

This literature review reveals several critical gaps that this dissertation addresses:

\textbf{Theoretical Integration Gap}: Despite clear theoretical connections, research on tolerance, social influence, and network interventions has developed largely independently. This dissertation provides the first comprehensive integration of these frameworks.

\textbf{Predictive Modeling Gap}: Most network intervention research tests strategies after implementation rather than developing predictive models to optimize strategies before deployment. This dissertation develops predictive models based on empirically validated mechanisms.

\textbf{Behavioral Measurement Gap}: Existing research often relies on self-reported attitudes rather than behavioral measures, making it difficult to assess real-world intervention effectiveness. This dissertation uses cooperation behavior as a key outcome measure.

\textbf{Mechanism Specification Gap}: While complex contagion principles are widely accepted, few studies specify the precise psychological mechanisms underlying these processes. This dissertation provides detailed mechanism specification based on Social Judgment Theory.

\textbf{German Context Gap}: Most tolerance intervention research focuses on North American contexts, with limited research in European contexts where tolerance challenges may differ. The German high school context provides important cross-cultural validation.

These gaps motivate the current research, which develops and tests a comprehensive theoretical framework linking tolerance attitudes to cooperation behaviors through social influence processes in friendship networks. The next chapter operationalizes this theoretical framework through formal mathematical models and testable hypotheses.