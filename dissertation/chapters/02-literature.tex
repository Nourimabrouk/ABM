%%%%%%%%%%%%%%%%%%%%%%%%%%%%%%%%%%%%%%%%%%%%%%%%%%%%%%%%%%%%%%%%%%%%%%%%%%%%%%%
% Chapter 2: Literature Review and Theoretical Foundation
% Agent-Based Models for Statistical Sociology
%%%%%%%%%%%%%%%%%%%%%%%%%%%%%%%%%%%%%%%%%%%%%%%%%%%%%%%%%%%%%%%%%%%%%%%%%%%%%%%

\chapter{Literature Review and Theoretical Foundation}
\label{chap:literature}

\section{Introduction}

This chapter provides a comprehensive review of the theoretical and methodological literature that forms the foundation for the ABM-RSiena integration framework developed in this dissertation. The review is organized around four interconnected bodies of literature: (1) agent-based modeling in social science, with particular attention to empirical validation challenges; (2) stochastic actor-oriented models and longitudinal network analysis; (3) tolerance, prejudice, and interethnic relations research; and (4) social influence and diffusion processes in networks. The integration of insights from these diverse literatures provides the theoretical foundation for the methodological innovations and empirical analyses presented in subsequent chapters.

The chapter begins with an examination of the evolution of agent-based modeling in social science, tracing its development from early theoretical applications to current methodological challenges. We then explore the parallel development of stochastic actor-oriented models and their contributions to understanding network-behavior co-evolution. The third section reviews the social psychological literature on tolerance and intergroup relations, establishing the theoretical foundation for the substantive application domain. The final section examines research on social influence and diffusion processes, providing insights about the mechanisms through which tolerance-based interventions might spread through social networks.

\section{Agent-Based Modeling in Social Science}

\subsection{Historical Development and Core Principles}

Agent-based modeling emerged in the social sciences during the 1990s as part of a broader computational turn that promised to transform how researchers study complex social phenomena \citep{gilbert1999computer, epstein1996growing}. The approach was initially motivated by limitations of traditional mathematical models that relied on strong assumptions about aggregation and equilibrium, often obscuring the micro-level processes that generate macro-level patterns \citep{tesfatsion2002agent}.

The foundational insight of agent-based modeling is that complex social phenomena can be understood as emergent properties of interactions among relatively simple individual agents \citep{epstein2006generative}. This perspective draws on complexity science and systems theory, emphasizing how local interactions can generate global patterns that cannot be predicted from knowledge of individual behavior alone \citep{miller2007complex}.

Early influential work by \citet{schelling1971dynamic} on residential segregation demonstrated the power of this approach by showing how mild preferences for same-race neighbors could generate extreme segregation patterns at the city level. This work established several key principles that continue to guide ABM research: (1) macro-level patterns can emerge from micro-level interactions; (2) individual rationality does not necessarily lead to collective rationality; and (3) small changes in individual behavior can have large aggregate effects.

The methodological foundations of ABM were further developed through work on artificial societies \citep{epstein1996growing}, computational economics \citep{tesfatsion2006handbook}, and social simulation \citep{gilbert2008agent}. These contributions established ABM as a legitimate scientific methodology with its own theoretical foundations, computational techniques, and standards for model development and evaluation.

\subsection{Applications in Social Science Research}

Over the past three decades, ABM has been applied to an enormous range of social science questions, spanning economics, sociology, political science, psychology, and anthropology. Some of the most influential applications include models of cooperation and collective action \citep{axelrod1997complexity}, norm emergence and diffusion \citep{bicchieri2006grammar}, organizational behavior \citep{march1991exploration}, political opinion formation \citep{deffuant2000mixing}, and social movement dynamics \citep{andrews2016strategies}.

In sociology specifically, ABM has been used to study fundamental questions about social integration, inequality, and social change. \citet{macy2002factors} identified several areas where ABM offers particular advantages over traditional approaches: situations involving spatial effects, network externalities, threshold effects, and path dependence. These characteristics are common in many sociological phenomena, making ABM a natural methodological choice for sociological research.

Notable sociological applications include \citet{hedstrom2005dissecting} work on analytical sociology, which uses ABM to develop and test mechanistic explanations for social phenomena. \citet{bearman2004chains} used ABM to study sexual networks and disease transmission, while \citet{watts2002simple} explored the conditions under which social influence processes lead to cultural convergence or polarization.

Recent work has extended ABM applications to new domains including social media dynamics \citep{bail2017emotional}, educational achievement gaps \citep{fiel2013decomposing}, and criminal behavior \citep{groff2007simulation}. These applications demonstrate the continued vitality of ABM as a research methodology and its potential for addressing contemporary social challenges.

\subsection{Methodological Challenges and Limitations}

Despite its theoretical appeal and empirical applications, ABM faces several significant methodological challenges that limit its scientific validity and policy relevance. The most fundamental challenge concerns empirical validation: how can researchers assess whether ABM predictions accurately represent real-world processes?

\citet{windrum2007empirical} identified several approaches to ABM validation, including indirect calibration (matching model outputs to aggregate statistics), direct calibration (using micro-level data to set parameter values), and out-of-sample prediction testing. However, they noted that most ABM research relies primarily on indirect calibration, which provides weak evidence for model validity because multiple different models can reproduce the same aggregate patterns.

\citet{fagiolo2019validation} further developed this critique, arguing that ABM research often suffers from "empirical poverty" – models are constructed based on theoretical assumptions rather than empirical evidence, and validation consists mainly of showing that models can reproduce stylized facts rather than conducting rigorous statistical tests. This approach makes it difficult to distinguish between models that capture real social mechanisms and those that merely provide convenient mathematical descriptions of observed patterns.

The parameter estimation problem in ABM research is particularly acute. Most ABM studies use parameter values that are chosen through informal calibration procedures or based on rough approximations from the literature. This approach makes it impossible to quantify uncertainty around parameter estimates or to conduct formal hypothesis tests about the strength of different mechanisms \citep{grazzini2012analysis}.

The temporal dynamics problem represents another significant challenge. ABM simulations typically operate in discrete time steps with arbitrary time units, while real social processes unfold in continuous time with measurable durations and rates. This disconnect makes it difficult to use longitudinal data to validate ABM assumptions or to make predictions about the timing of intervention effects \citep{lorscheid2012opening}.

\subsection{Recent Advances in ABM Methodology}

Recent methodological research has begun to address some of these limitations through several promising approaches. Bayesian methods for ABM parameter estimation have been developed to enable formal statistical inference about agent behaviors \citep{beaumont2002approximate, sisson2007sequential}. These methods use simulation-based techniques to estimate posterior distributions over parameter values, enabling uncertainty quantification and hypothesis testing.

Machine learning approaches have been applied to ABM calibration and validation, including the use of genetic algorithms for parameter optimization \citep{calvez2005agent} and neural networks for pattern recognition in model outputs \citep{hassan2008agent}. These techniques can handle the high-dimensional parameter spaces and complex output patterns that characterize many ABM applications.

Network-based ABM has emerged as a particularly active area of methodological development, with researchers developing new approaches for incorporating realistic network structures into agent interactions \citep{jackson2008social}. These developments are particularly relevant for the current research, as they provide the foundation for integrating ABM with network analysis methodologies.

\section{Stochastic Actor-Oriented Models and Network Analysis}

\subsection{Theoretical Foundations of SAOM}

Stochastic Actor-Oriented Models represent a fundamentally different approach to modeling network-behavior co-evolution, one that prioritizes statistical rigor and empirical validation over simulation flexibility \citep{snijders2001statistical}. The theoretical foundation of SAOM rests on several key principles that distinguish it from other network modeling approaches.

The actor-oriented perspective assumes that network change results from the decisions of individual actors who have agency in forming, maintaining, and dissolving social ties \citep{snijders2010introduction}. This perspective contrasts with tie-oriented models that treat relationships as the primary units of analysis, and with models that assume network structure is determined by exogenous factors rather than endogenous choices.

The stochastic component of SAOM acknowledges that social behavior involves uncertainty and that researchers cannot predict individual decisions with certainty, even with complete information about relevant factors. Instead, SAOM models specify probability distributions over possible behaviors, allowing for statistical inference about the factors that influence these probabilities \citep{snijders2017stochastic}.

The continuous-time assumption in SAOM provides a more realistic representation of social processes than discrete-time alternatives. In continuous time, actors can potentially make changes at any moment, with the timing of changes determined by stochastic processes. This approach enables more accurate modeling of the temporal dynamics of network evolution \citep{snijders2001statistical}.

\subsection{Mathematical Framework and Estimation Procedures}

The mathematical framework underlying SAOM is based on continuous-time Markov chain models that describe how networks and behaviors evolve through sequences of small changes \citep{snijders2010introduction}. At each moment in time, one actor is selected to have the opportunity to make a change, with the probability of selection determined by rate functions that can depend on actor characteristics and network position.

When an actor has the opportunity to change, they evaluate alternative network configurations (or behavior values) according to an objective function that captures their preferences. The objective function is typically specified as a linear combination of network statistics, with coefficients (parameters) that must be estimated from data. The probability of choosing a particular alternative follows a multinomial logit model based on the objective function values.

The co-evolution of networks and behaviors is modeled through separate but interrelated processes. Network evolution is governed by objective functions that can depend on both network structure and actor behaviors, while behavior evolution is governed by objective functions that can depend on network position and the behaviors of network neighbors. This framework naturally captures both selection effects (network effects on behavior) and influence effects (behavior effects on network formation).

Parameter estimation in SAOM is based on the method of moments, implemented through simulation procedures that approximate the likelihood function \citep{snijders2002markov}. The estimation algorithm iteratively adjusts parameter values until the expected values of sufficient statistics under the model match their observed values in the data. Standard errors are computed using the observed information matrix, enabling conventional statistical inference procedures.

\subsection{Empirical Applications and Validation}

SAOM methodology has been applied to a wide range of substantive questions in social network analysis, with applications spanning friendship formation \citep{steglich2010dynamic}, organizational networks \citep{rank2010structural}, international relations \citep{cranmer2012complex}, and health behaviors \citep{mercken2010dynamics}. These applications have demonstrated the value of the SAOM approach for understanding network-behavior co-evolution in real social contexts.

One of the key advantages of SAOM is its ability to handle missing data and measurement error in systematic ways that preserve the validity of statistical inference. The method can accommodate incomplete network data, missing behavioral measurements, and uncertainty about the timing of changes \citep{huisman2008treatment}. This robustness is particularly important for longitudinal network studies, which often face significant data collection challenges.

The validation of SAOM models involves several types of assessments, including goodness-of-fit tests for specific network statistics, overall model adequacy checks, and out-of-sample prediction evaluations \citep{lospinoso2019goodness}. These validation procedures provide much stronger evidence for model validity than the informal approaches typically used in ABM research.

Recent methodological developments in SAOM include extensions to multiplex networks \citep{snijders2013multilevel}, meta-analytic approaches for combining results across studies \citep{schweinberger2020settings}, and Bayesian estimation procedures \citep{koskinen2013bayesian}. These developments continue to expand the applicability and rigor of the SAOM approach.

\subsection{Limitations and Challenges}

Despite its methodological strengths, SAOM also faces several limitations that restrict its applicability for certain research questions. The continuous-time Markov assumption may not be appropriate for all social processes, particularly those involving memory effects, learning, or strategic behavior that extends over longer time horizons \citep{block2018forms}.

The computational complexity of SAOM estimation procedures limits the size of networks that can be analyzed and the complexity of behavioral mechanisms that can be incorporated. Large networks (>200 actors) require specialized computational approaches, and complex behavioral models may not converge during estimation \citep{ripley2021manual}.

The focus on statistical inference in SAOM makes it difficult to explore counterfactual scenarios or to simulate intervention effects. While SAOM provides excellent tools for understanding observed processes, it offers limited capabilities for exploring "what if" questions that are central to policy analysis and intervention design \citep{duxbury2021logic}.

The assumption that network evolution is driven by individual actor decisions may not capture all relevant social mechanisms. Some network changes may result from external constraints, institutional rules, or collective processes that are not well-represented by actor-oriented models \citep{lewis2008social}.

\section{Tolerance, Prejudice, and Interethnic Relations}

\subsection{Theoretical Foundations}

The study of interethnic relations has a long history in social psychology and sociology, with theoretical frameworks evolving from early work on prejudice and stereotyping to more nuanced understandings of the multiple motivations that shape intergroup behavior \citep{allport1954nature, pettigrew1998intergroup}. Contemporary research recognizes that interethnic relations are influenced by multiple psychological and social processes that operate at different levels of analysis.

The distinction between prejudice and tolerance has emerged as a central organizing principle in recent theoretical work \citep{verkuyten2023tolerance}. Prejudice refers to negative affect, stereotyping, and discrimination directed toward members of ethnic outgroups based on their group membership. Tolerance, by contrast, refers to the acceptance of beliefs, values, or practices despite disapproval or disagreement \citep{pain2021tolerance}.

This distinction is important because it recognizes that individuals can disapprove of outgroup practices for principled reasons (such as commitment to gender equality) while still choosing to tolerate those practices out of respect for diversity or democratic values. Traditional interventions that focus solely on reducing prejudice may miss this important source of intergroup tension and fail to promote the kind of respectful coexistence that tolerance represents.

\subsection{Social Psychological Mechanisms}

Research in social psychology has identified several key mechanisms that influence tolerance and prejudice. Contact theory, originally developed by \citet{allport1954nature} and refined by \citet{pettigrew1998intergroup}, suggests that positive intergroup contact can reduce prejudice under certain conditions: equal status, common goals, intergroup cooperation, and institutional support.

However, recent research has shown that contact effects may not automatically extend to tolerance, particularly when contact occurs in contexts where group differences remain salient \citep{dovidio2003intergroup}. Contact may reduce affective prejudice while leaving cognitive disapproval unchanged, suggesting that different intervention strategies may be needed to promote tolerance.

Social identity theory provides another important theoretical framework for understanding interethnic relations \citep{tajfel1979integrative}. This theory suggests that individuals derive part of their self-concept from membership in social groups and are motivated to maintain positive distinctiveness for their ingroups. These motivations can lead to ingroup favoritism and outgroup derogation, even in the absence of realistic conflict.

The social identity approach has been extended to understand tolerance through the concept of inclusive identity, where individuals develop broader, more encompassing group identities that include multiple ethnic groups \citep{gaertner2000reducing}. Interventions that promote inclusive identity may be more effective at fostering tolerance than those that simply try to reduce the salience of ethnic categories.

\subsection{Network Effects and Social Influence}

Recent research has increasingly recognized the importance of social networks for understanding how tolerance and prejudice develop and change over time. Social influence processes within friendship networks can lead to convergence in intergroup attitudes, with friends becoming more similar in their levels of prejudice and tolerance over time \citep{boda2020short}.

However, the mechanisms underlying these influence processes are complex and may vary depending on the specific attitudes involved and the characteristics of the individuals and relationships. Some research suggests that influence on tolerance may follow different patterns than influence on prejudice, with tolerance being more susceptible to reasoned persuasion and less susceptible to emotional contagion \citep{simon2023tolerance}.

The structure of social networks also influences the effectiveness of tolerance interventions. Networks that are highly segregated by ethnicity may provide fewer opportunities for cross-group contact and may be more resistant to change in intergroup attitudes \citep{moody2001race}. Conversely, networks with high levels of cross-group friendship may facilitate the spread of tolerant attitudes through influence processes.

Network interventions that strategically target influential individuals or that attempt to create new cross-group connections may be more effective than interventions that focus solely on changing individual attitudes \citep{paluck2011peer}. This insight provides a theoretical foundation for the network-based intervention strategies examined in this dissertation.

\subsection{Intervention Research}

Empirical research on tolerance interventions has grown significantly in recent years, with studies examining various approaches including perspective-taking exercises, multicultural education, intergroup dialogue, and contact-based programs \citep{lemmer2015can}. However, most intervention research has focused on individual-level change rather than network-level diffusion processes.

The study by \citet{shani2023tolerance} represents one of the few attempts to implement and evaluate a tolerance intervention designed to spread through social networks. While the intervention showed limited effectiveness in the original study, the rich longitudinal data collected as part of the research provides valuable insights about the conditions under which tolerance interventions might succeed or fail.

Meta-analytic research on prejudice reduction interventions has identified several factors that influence intervention effectiveness, including the duration and intensity of the intervention, the characteristics of participants, and the social context in which the intervention is implemented \citep{paluck2019prejudice}. However, much less is known about the factors that influence the spread of intervention effects through social networks.

\section{Social Influence and Diffusion Processes}

\subsection{Theoretical Models of Social Influence}

Social influence research has developed several theoretical models that describe how attitudes, behaviors, and innovations spread through social networks. These models differ in their assumptions about the mechanisms underlying influence processes and the conditions under which influence occurs.

The social learning model suggests that individuals adopt new attitudes or behaviors through observation and imitation of others, particularly those who are similar to themselves or who occupy prestigious positions \citep{bandura1977social}. This model emphasizes the role of modeling and vicarious reinforcement in social influence processes.

The social impact model, developed by \citet{latane1981psychology}, proposes that social influence is a function of the strength, immediacy, and number of influence sources. Strength refers to the importance or credibility of the source, immediacy refers to proximity in space and time, and number refers to the quantity of sources. This model has been applied to understand conformity, attitude change, and collective behavior.

Social judgment theory provides a more nuanced understanding of attitude change processes, suggesting that individuals have latitude of acceptance, rejection, and non-commitment around their current attitudes \citep{sherif1961social}. Influence attempts that fall within the latitude of acceptance are likely to be successful, while those that fall within the latitude of rejection may produce boomerang effects.

\subsection{Simple vs. Complex Contagion}

A major theoretical distinction in diffusion research concerns the difference between simple and complex contagion processes \citep{centola2010spread}. Simple contagion refers to processes where exposure to a single activated individual is sufficient for adoption, while complex contagion refers to processes where multiple exposures or social reinforcement are required.

Simple contagion models are appropriate for phenomena like disease transmission or information diffusion, where a single contact can be sufficient for transmission. Complex contagion models are more appropriate for phenomena like behavior change or belief adoption, where social proof and legitimacy concerns make multiple exposures necessary.

Research by \citet{centola2010spread} has shown that network structure affects simple and complex contagion processes differently. Simple contagion spreads most effectively through networks with long-range connections and low clustering, while complex contagion spreads most effectively through networks with high clustering and strong local connectivity.

The distinction between simple and complex contagion has important implications for tolerance interventions. If tolerance change follows a complex contagion process, then interventions may be more effective when they target clustered groups of individuals rather than dispersed individuals, and when they create multiple sources of social reinforcement for tolerance.

\subsection{Network Structure and Diffusion}

The structure of social networks plays a crucial role in determining how influence processes unfold and which intervention strategies are most effective. Different network structures create different opportunities and constraints for the spread of attitudes and behaviors.

Small-world networks, characterized by high clustering and short path lengths, may be particularly conducive to complex contagion processes because they provide both the local reinforcement needed for adoption and the global connectivity needed for rapid spread \citep{watts1998collective}. However, the effectiveness of small-world structures may depend on the specific mechanisms underlying the diffusion process.

Scale-free networks, characterized by highly skewed degree distributions with a few highly connected nodes, may be vulnerable to targeted interventions that focus on high-degree individuals \citep{barabasi2002linked}. Removing or influencing a small number of hubs can have disproportionate effects on network connectivity and diffusion processes.

Network segregation and homophily can create barriers to diffusion by limiting contact between different groups and reinforcing existing divisions \citep{mcpherson2001birds}. Interventions that aim to overcome these barriers may need to explicitly address network structure rather than simply targeting individual attitudes.

\subsection{Strategic Intervention Design}

Recent research has begun to develop strategic approaches to intervention design that leverage network structure and influence processes for maximum effectiveness. These approaches recognize that the success of interventions depends not only on their content but also on their implementation strategy within social networks.

Targeting strategies based on network centrality measures have received considerable attention, with research examining the relative effectiveness of targeting high-degree individuals, brokers who connect different network regions, and peripheral individuals who may be more susceptible to influence \citep{valente2012network}. The optimal targeting strategy appears to depend on the characteristics of the diffusion process and the structure of the network.

Timing strategies that coordinate intervention delivery to take advantage of network effects have also been explored. Sequential interventions that begin with influential individuals and gradually expand to their network neighbors may be more effective than simultaneous interventions that target many individuals at once \citep{garcia2020engineering}.

The development of these strategic approaches represents an important advance in intervention research, moving beyond simple individual-level interventions toward more sophisticated network-based approaches that can leverage social influence processes for greater effectiveness.

\section{Integration and Synthesis}

\subsection{Convergent Insights Across Literatures}

The review of these four bodies of literature reveals several convergent insights that provide the foundation for the methodological integration pursued in this dissertation. First, all four literatures recognize the importance of social networks for understanding individual behavior and social change. Whether the focus is on agent interactions in ABM, network-behavior co-evolution in SAOM, intergroup contact in tolerance research, or diffusion pathways in influence research, social networks emerge as a central organizing principle.

Second, all four literatures grapple with questions about the mechanisms underlying social influence processes. ABM research focuses on specifying behavioral rules that govern agent interactions; SAOM research focuses on estimating statistical models that capture influence effects; tolerance research focuses on understanding the psychological processes that mediate attitude change; and diffusion research focuses on identifying the conditions under which influence occurs.

Third, all four literatures recognize the importance of temporal dynamics for understanding social processes. ABM simulations unfold over time through sequences of agent interactions; SAOM models capture the continuous-time evolution of networks and behaviors; tolerance interventions aim to create lasting change in intergroup attitudes; and diffusion processes involve the sequential spread of innovations through populations.

\subsection{Complementary Strengths and Limitations}

The review also reveals that these different approaches offer complementary strengths that, when combined, can address the limitations of individual approaches. ABM provides flexible tools for exploring theoretical possibilities and simulating intervention scenarios, but lacks rigorous empirical validation procedures. SAOM provides sophisticated statistical inference capabilities, but offers limited tools for exploring counterfactual scenarios.

Tolerance research provides detailed understanding of the psychological mechanisms underlying attitude change, but has limited tools for modeling network-level diffusion processes. Diffusion research provides insights about the conditions under which influence processes succeed or fail, but often lacks detailed psychological foundations for understanding the mechanisms involved.

The integration of these approaches offers the possibility of combining their respective strengths while mitigating their individual limitations. ABM can provide the simulation capabilities needed to explore intervention scenarios, while SAOM can provide the statistical rigor needed for empirical validation. Tolerance research can provide the psychological foundations for understanding attitude change mechanisms, while diffusion research can provide insights about optimal intervention strategies.

\subsection{Theoretical Framework for Integration}

The theoretical framework that emerges from this literature review suggests that tolerance interventions can be understood as complex contagion processes that operate through social networks via mechanisms of social influence and social selection. This framework integrates insights from all four literatures into a coherent theoretical perspective that can guide both methodological development and empirical analysis.

The complex contagion perspective suggests that tolerance change requires multiple exposures and social reinforcement, making network structure and intervention strategy crucial determinants of success. The social influence mechanism suggests that individuals change their tolerance levels in response to the attitudes and behaviors of their network neighbors, with the likelihood and direction of change depending on factors such as attitude similarity, relationship strength, and social identity.

The social selection mechanism suggests that tolerance levels influence individuals' decisions about forming and maintaining social relationships, with implications for the evolution of network structure over time. The intervention perspective suggests that external attempts to change tolerance levels can be understood as perturbations to these natural influence and selection processes.

This integrated theoretical framework provides the foundation for the methodological development and empirical analysis presented in subsequent chapters. It specifies the key mechanisms that must be modeled, the types of data needed for empirical validation, and the intervention strategies that should be tested through simulation studies.

\section{Conclusion}

This literature review has provided a comprehensive examination of the theoretical and methodological foundations for the ABM-SAOM integration framework developed in this dissertation. The review reveals that while each individual approach has important limitations, their integration offers the possibility of significant methodological advances that can address key challenges in computational social science research.

The substantive focus on tolerance interventions provides an ideal application domain for demonstrating the value of methodological integration. The complex mechanisms underlying tolerance diffusion, the availability of high-quality longitudinal network data, and the policy relevance of intervention effectiveness research create strong incentives for developing more rigorous methodological approaches.

The theoretical framework that emerges from the literature review provides clear guidance for the methodological development and empirical analysis presented in subsequent chapters. This framework specifies tolerance interventions as complex contagion processes operating through networks via influence and selection mechanisms, with effectiveness depending on network structure, intervention strategy, and psychological mechanisms of attitude change.

The next chapter builds on this theoretical foundation to develop a formal methodological framework that integrates ABM and SAOM approaches. This framework operationalizes the theoretical insights reviewed in this chapter and provides the technical foundation for the empirical analyses that follow.